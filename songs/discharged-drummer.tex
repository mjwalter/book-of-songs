\songtitle[The ]{discharged drummer}{\narrative\\\love}

\notes{f \major}{3/2}{\partial 4 r8 f f[ f] f8.[ g16] f4 d2 r8 c d'8.[ d16] c8[ c] a2.}\contd
\notescontd{f \major}{3/2}{\partial 4 r8 f d'8[ d8] c8[ c] a4\turn g2 r8 g8 f8[ a8] g16[ a8.] f2. \bar "||"}

\versemark
In Bristol lived a lady\\*
She was scarce sixteen\\
Courted she was by many\\*
Her favor for to win

\versemark
But none of them could suit her\\*
Or please her to her mind\\
Until there came a drummer\\*
So loving and so kind

\versemark
Well the drummer he stepped up to her\\*
And he stole from her one kiss\\
He said, Dear honoured lady\\*
In the regiment will you enlist?

\versemark
— Oh yes, replied the lady\\*
This I’ll surely do\\
For while I love your music sweet\\*
Likewise your rattatoo

\versemark
And if you consent to marry\\*
Or to lie by my side,\\
I’ll buy your discharge my love\\*
In a carriage you will ride

\versemark
— Oh yes, replied the drummer\\*
How happy I should be,\\
But I’m afraid that you won’t lie\\*
With such a poor man as me

\versemark
— Oh that you may depend on,\\*
The lady made reply,\\
For if I don’t wed with you, young man,\\*
I’ll never be made a bride

\versemark
This couple they got married\\*
With servants at their call,\\
And the drummer’s left off playing\\*
Among his comrades all

\versemark
But sometimes when the moon is high\\*
This drummer steals away,\\
And plays on his old army drum\\*
Till the dawning of the day

\attribution{Roud 2303; Nightingale (group), performer}
